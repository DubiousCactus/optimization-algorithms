\section{Introduction}
The importance of data analysis and machine learning in our modern day world is increasing day by day.
And so is the number of different algorithms used.
In this paper, five of the most basic classification algorithms are used to solve an images classification problem on two very popular datasets: The MNIST dataset of handwritten numbers and the ORL dataset of faces.
Using the corresponding labels, each algorithm is trained in a supervised training process and then applied on a test set.
The resulting success rate in combination with the execution time will be a measure on how efficient each algorithm is.

In addition to applying the algorithms on the raw data, they are also applied on a reduced version of the dataset.
This reduced version is obtained by applying the common \textit{Principal Component Analysis}.

The following algorithms have been choosen and implemented in \texttt{Matlab}:
\begin{itemize}
	\item Nearest Centroid Classifier
	\item Nearest Subclass Centroid Classifier
	\item Nearest Neighbor Classifier
	\item Perceptron, trained with Backpropagation
	\item Perceptron, trained using the Minimum Square Error
\end{itemize}

The MNIST dataset consists of 60000 images while the ORL dataset only consists of 400.
This opposes a problem since some of the algorithms rely on a large set of parameters that have to be optimized during the supervised learning.
To overcome this problem and still get sufficient results, each algorithm is applied multiple times and the success rates and execution times are averaged.
This helps to make the overall results more predictable.

